\documentclass{article}
\usepackage{amsmath,amsthm,amssymb}
\usepackage{xltxtra}
%\usepackage[left=1in,right=0.8in,top=1.2in,bottom=0.9in]{geometry}

\setmainfont{STKaiti}
\setsansfont{STHeiti}

\title{斜级数环中的代数运算}
%\author{lxj}
\date{}
\newtheorem{lem}{引理}
\newtheorem{prop}{命题}
\newtheorem{thm}{定理}

\begin{document}
\maketitle

\XeTeXlinebreaklocale "zh"
\XeTeXlinebreakskip = 0pt plus 1pt

{\tt SkewSeries} 是一个范畴 ({\tt Category}),用数学的语言表示为
\begin{equation}
  \label{eq:1}
  \mathtt{SkewSeries}(R,\sigma,\delta) := \left\{ \sum_{i=N}^{+\infty}a_ix^{-i}\bigg| a_i\in R\right\}
\end{equation}
其中 $x$ 是未定元,$R$ 是一个含幺环,$\sigma:R\longrightarrow R$ 是 $R$ 上的一个自同构,$\delta:
R\longrightarrow R$ 是一个与$\sigma$相联系的导子,即满足$\delta(ab) = \delta(a) b + \sigma(a)
\delta(b)$, 对环中任意元素$a$,$b$. 所谓斜级数,是指 $x$ 与 $a\in R$ 的乘法是不可交换的,满足
\begin{equation}
  \label{eq:2}
  x a = \sigma(a) x + \delta(a)
\end{equation}
从这个最基本的运算法则出发,我们可以定义$x^{-1}a$:假设$x^{-1} a = \sum_{i=1}^{+\infty}a_i x^{-i}$,
利用待定系数法,我们知道
\begin{align*}
  x\cdot x^{-1} a = a &= x\cdot \sum_{i=1}^{+\infty}a_i x^{-i}\\
  &=\sigma(a_1) + \sum_{i=1}^{+\infty}\left(\sigma(a_{i+1})+\delta(a_i)\right)x^{-i}
\end{align*}
比较各次幂的系数,我们得到
\begin{align*}
  a_1 &= \sigma^{-1}(a)\\
  a_2 &= -\sigma^{-1}\delta(a_1)\\
  a_3 &= -\sigma^{-1}\delta(a_2)=\left(-\sigma^{-1}\delta\right)^2\sigma^{-1}(a)\\
  \cdots &=\cdots\\
  a_{i+1} &= \left(-\sigma^{-1}\delta\right)^{i}\sigma^{-1}(a).
\end{align*}
因此
\begin{displaymath}
  x^{-1} a = \sum_{i=1}^\infty \left(-\sigma^{-1}\delta\right)^{i-1}\sigma^{-1}(a) x^{-i}.
\end{displaymath}


\begin{lem}\fontspec{Hei}
  设一个最高项为$N$的斜级数$$s=a_1 x^N+a_2 x^{N-1} + a_3 x^{N-2}+ \cdots $$
  则$x^{-1} s = \sum_{k=1}^{+\infty} b_k x^{N-k}$, 其中
  \begin{displaymath}
    b_k = \sum_{i=1}^k(-\sigma^{-1}\delta)^{k-i}\sigma^{-1}(a_i).
  \end{displaymath}
\end{lem}
\begin{proof}[证明.]
  \begin{align*}
    x^{-1} s &=  x^{-1} \sum_{i=1}^{+\infty}a_ix^{-i}\cdot x^{N+1} \\ 
    &= \sum_{i=1}^\infty \sum_{j=1}^\infty (-\sigma^{-1}\delta)^{j-1}\sigma^{-1}(a_i) x^{-i-j}\cdot
    x^{N+1}\\
    &=\sum_{k=2}^\infty\sum_{i=1}^{k-1}(-\sigma^{-1}\delta)^{k-i-1} \sigma^{-1}(a_i) x^{N+1-k}\\
    &=\sum_{k=1}^\infty\sum_{i=1}^k (-\sigma^{-1}\delta)^{k-i}\sigma^{-1}(a_i) x^{N-k}
  \end{align*}
\end{proof}
而易知
\begin{displaymath}
  x s = \sigma(a_N)x^{-N+1}+\sum_{j=N}^{+\infty}\left(\sigma(a_{j+1})+\delta(a_j)\right) x^{-j}.
\end{displaymath}

\section{接口}


\end{document}
